\documentclass{article}

\usepackage{magicvariables}
\makeVar{x}{noFmt}{x}
\makeVari{xx}{noFmt}{x}
\makeVar{boldX}{mathbf}{x}
\makeVari{boldXx}{mathbf}{x}
\makeVar{sfX}{textsf}{x}
\makeVari{sfXx}{textsf}{x}

\title{Magic Variables}
\author{Fangyi Zhou}
\begin{document}
\maketitle
\section{Usage}

\verb+\makeVar{x}{noFmt}{x}+ defines the macro \verb+\x+ with \verb+\noFmt+
formatting macro, with content \verb+x+.
\verb+\makeVar{boldX}{mathbf}{x}+ defines the macro \verb+\boldX+ with
\verb+\mathbf+ formatting macro, with content \verb+x+.

$\x$ is a normal x, $\boldX$ is a bold x.

$\xx$ is a normal x prime, $\boldXx$ is a bold x prime.

\verb+\makeVari{xx}{noFmt}{x}+ defines the macro \verb+\xx+ with \verb+\noFmt+
formatting macro, with content \verb+x+, with a prime superscript.
\verb+\makeVari{boldXx}{mathbf}{x}+ defines the macro \verb+\boldXx+ with
\verb+\mathbf+ formatting macro, with content \verb+x+, with a prime
superscript.

$\x[0]$ is a normal x with subscript, $\boldX[0]$ is a bold x with subscript.

$\xx[0]$ is a normal x prime with subscript, $\boldXx[0]$ is a bold x prime
with subscript.

Note: the macros still works if a text formatting macro is used, e.g.\
\verb+\makeVar{sfX}{textsf}{x}+ uses a text formatting macro.

$\sfX$ is a serif x, $\sfXx$ is a serif x prime.

$\sfX[i]$ is a serif x with subscript, $\sfXx[i]$ is a serif x prime with
subscript.
The subscript is rendered in maths font instead of the text formatting macro.

\section*{Acknowledgement}

Thanks Alceste Scalas for the original macro design, the author simply added
automation in making the variables.

\end{document}
